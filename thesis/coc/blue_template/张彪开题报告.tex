%!TEX program = xelatex
\documentclass[10pt,mathserif]{beamer}%,aspectratio=169 %画面比例16:9%
\usepackage{biblatex}
\usepackage{xmpmulti}
\usepackage{xeCJK}
\usepackage{hyperref} % links
\usepackage{pgfgantt}
\usetheme[
 sidebar, xdblue
 ]{XDUstyle}
\usepackage{mdframed} 
\theoremstyle{definition}
\mdfdefinestyle{theoremstyle}{%
	linecolor=gray!40,linewidth=.5pt,%
	backgroundcolor=gray!10,
	skipabove=8pt,
	skipbelow=5pt,
	innerleftmargin=7pt,
	innerrightmargin=7pt,
	frametitlerule=true,%
	frametitlerulewidth=.5pt,
	frametitlebackgroundcolor=gray!15,
	frametitleaboveskip=0pt,
	frametitlebelowskip=0pt,
	innertopmargin=.4\baselineskip,
	innerbottommargin=.4\baselineskip,
	shadow=true,shadowsize=3pt,shadowcolor=black!20,
	theoremseparator={.},
}
%[section]means add the section number in the first place.
\mdtheorem[style=theoremstyle]{defi}{\hspace{0em}$\blacklozenge$定义}
\mdtheorem[style=theoremstyle]{solu}{\hspace{0em}$\checkmark$思路}

\title{张彪开题报告标题}
\subtitle{子标题}
\institute[中国科学院大学\\生命科学学院]{指导老师:王艳芬} % 中括号部分为导航栏底所用尽可能精简
\author{张彪}
\date{2016年07月05日}% 时间可自行设置
	
\begin{document}%
{\xdbg \frame[plain,noframenumbering]{\titlepage}}%首页标题页

%\part{第一部分}

\section{背景}

\begin{frame}
	背景介绍 无需多说
\end{frame}

\begin{frame}{a}{b}
	背景介绍
	\vspace{1em}
	\begin{itemize}
		\item 条目1 \footfullcite{http://www.marketingcharts.com/online/}, 65\% of viewers: recommendation, while only 57\% of viewers:  browsing.
		\item 条目2 \footfullcite{Amazon's recommendation secret 2012}.
	\end{itemize}
\end{frame}


\section{第一部分}
\begin{frame}{手把手教你画猫}
	\setbeamercovered{invisible}
	\multiinclude[format=jpg,start=1,graphics={width=.9\textwidth}]{aid1312391-900px-Draw-a-Cat-Face-Step}
\end{frame}
\subsection{问题定义}
\begin{frame}{a}
	\begin{defi}
		巴拉巴拉我在 这里修改了....
	\end{defi}
\end{frame}
\subsection{解决思路}
\begin{frame}{a}
	\begin{solu}
		思路一
	\end{solu}
	\vspace{1.4cm}
	巴拉巴拉
\end{frame}

\begin{frame}{a}	
	\setcounter{solu}{1}
	\begin{solu}
		思路二
	\end{solu}
	\vspace{1.4cm}
	balabal巴拉巴拉
\end{frame}

\section{第二部分}
\
\subsection{问题定义}
\begin{frame}{a}
	\begin{defi}
		巴拉巴拉
	\end{defi}
\end{frame}
\subsection{解决思路}
\begin{frame}{a}
	\begin{solu}
		思路一
	\end{solu}
	\vspace{1.4cm}
	巴拉巴拉
\end{frame}

\begin{frame}{a}	
	\setcounter{solu}{1}
	\begin{solu}
		思路二
	\end{solu}
	\vspace{1.4cm}
	balabal巴拉巴拉
\end{frame}

\section{第三部分}
\
\subsection{问题定义}
\begin{frame}{a}
	\begin{defi}
		巴拉巴拉
	\end{defi}
\end{frame}
\subsection{解决思路}
\begin{frame}{a}
	\begin{solu}
		思路一
	\end{solu}
	\vspace{1.4cm}
	巴拉巴拉
\end{frame}

\begin{frame}{a}	
	\setcounter{solu}{1}
	\begin{solu}
		思路二
	\end{solu}
	\vspace{1.4cm}
	balabal巴拉巴拉
\end{frame}

\section{总结}
\begin{frame}{a}
	我要做的工作如下: 假装做了而已
	\vspace{1em}
	\begin{itemize}
		\setlength\itemsep{1em}
		\item 第一
		\item 第二
		\item 第三
	\end{itemize}	
\end{frame}
\begin{frame}
	\begin{figure}
		\vspace{2em}
		\centering
		\resizebox{1.0\textwidth}{!}{\ganttset{group/.append style={orange},
	milestone/.append style={black},
	progress label node anchor/.append style={text=red}}

\begin{ganttchart}[%Specs
	y unit title=0.5cm,
	y unit chart=0.5cm,
	vgrid,hgrid,
	title height=1,
	%     title/.style={fill=none},
	title label font=\bfseries\footnotesize,
	bar/.style={fill=gray},
	%  bar incomplete/.style={fill=white},
	bar height=0.68,
	progress label text={},
	group right shift=0,
	group top shift=0.7,
	group height=.5,
	group peaks width={0.5},
	inline]{1}{14}
	%labels
	\gantttitle{时间计划表}{14}\\  % title 1
	%%
	\gantttitle[]{2015--2016}{4}
	\gantttitle[]{2016--2017}{4}                 % title 2
	\gantttitle[]{2017--2018}{4} 
	\gantttitle[]{2018}{2}
	\\ 
	%%
	\gantttitle{Q1}{1}                      % title 3
	\gantttitle{Q2}{1}
	\gantttitle{Q3}{1}
	\gantttitle{Q4}{1}
	%%  
	\gantttitle{Q1}{1}                      % title 3
	\gantttitle{Q2}{1}
	\gantttitle{Q3}{1}
	\gantttitle{Q4}{1}
	%%   
	\gantttitle{Q1}{1}                      % title 3
	\gantttitle{Q2}{1}
	\gantttitle{Q3}{1}
	\gantttitle{Q4}{1}
	%%    
	\gantttitle{Q1}{1}                      % title 3
	\gantttitle{Q2}{1}
	% \gantttitle{Q3}{1}
	% \gantttitle{Q4}{1}
	\\
	%% 
	\ganttbar[progress=67,inline=false]{阅读文献}{1}{6}\\ 
	\ganttmilestone[inline=false]{\underline{\color{blue}开题报告}}{4}\\
	\ganttbar[progress=0,inline=false]{历史数据Meta分析}{5}{6}\\
	
	\ganttbar[progress=0,inline=false]{野外补样}{7}{7} \\
	%% 
	\ganttbar[progress=0,inline=false]{撰写植多样性与微生物多样性文章}{8}{10} \\
	\ganttmilestone[inline=false]{\underline{\color{blue}中期答辩}}{10}\\
	\ganttbar[progress=0,inline=false]{博士学位论文撰写}{11}{12} \\
	\ganttmilestone[inline=false]{\underline{\color{blue}毕业答辩}}{13}	
\end{ganttchart}}
	\end{figure}
\end{frame}


{\xdbg%末页致谢
\begin{frame}[plain,noframenumbering]
 \finalpage{{\huge 感谢观看!}}
\end{frame}}


\end{document} 