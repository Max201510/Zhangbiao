%!TEX program = xelatex
\documentclass[10pt,mathserif]{beamer}%,aspectratio=169 %画面比例16:9%
\usepackage{verbatimbox,color}
\usepackage{picinpar,graphicx}
\usepackage{smartdiagram}
\usesmartdiagramlibrary{additions}
\usepackage{multicol}
\usepackage{biblatex}
\usepackage{xmpmulti}
\usepackage{xeCJK}
\usepackage{hyperref} % links
\usepackage{pgfgantt}
\usepackage{pdfpages}
\usetheme[
 sidebar, xdblue
 ]{XDUstyle}
\usepackage{mdframed} 
\theoremstyle{definition}
\mdfdefinestyle{theoremstyle}{%
	linecolor=gray!40,linewidth=.5pt,%
	backgroundcolor=gray!10,
	skipabove=8pt,
	skipbelow=5pt,
	innerleftmargin=7pt,
	innerrightmargin=7pt,
	frametitlerule=true,%
	frametitlerulewidth=.5pt,
	frametitlebackgroundcolor=gray!15,
	frametitleaboveskip=0pt,
	frametitlebelowskip=0pt,
	innertopmargin=.4\baselineskip,
	innerbottommargin=.4\baselineskip,
	shadow=true,shadowsize=3pt,shadowcolor=black!20,
	theoremseparator={.},
}
%[section]means add the section number in the first place.
\mdtheorem[style=theoremstyle]{defi}{\hspace{0em}$\blacklozenge$定义}
\mdtheorem[style=theoremstyle]{solu}{\hspace{0em}$\checkmark$思路}

\title{高寒草地和温性草地植物多样性和微生物多样性沿水分梯度变化的对比研究}
%\subtitle{微生物生态学}
\institute[中国科学院大学\\生命科学学院]{指导老师:王艳芬} % 中括号部分为导航栏底所用尽可能精简
\author{张彪}
\date{2016年07月05日}% 时间可自行设置
	
\begin{document}%
{\xdbg \frame[plain,noframenumbering]{\titlepage}}%首页标题页

%\part{第一部分}

\section{选题背景及研究现状}

\subsection{选题背景}
\begin{frame}{\insertsection}{\insertsubsection}
	\begin{center}
		\includegraphics[width = 0.9\textwidth]{./pic/中国草地资源.jpg}
	\end{center}
\end{frame}
\begin{frame}{Fixing an Optical Mouse}
	\begin{columns}
		\column{0.7\textwidth}
		\hskip 15ex 中国草地资源
		\begin{center}
			\includegraphics[width = 0.9\textwidth]{./pic/中国草地资源.jpg}
		\end{center}
		%\vspace{15mm}
		\only<6>{
		我国是世界上第二大草地资源大国,天然草原面积3.94×108hm2,约占我国国土面积的41.7\%,是耕地面积的3 倍左右,林地面积的2 倍多(刘黎明,2002)。
	    }
		\column{0.3\textwidth}
		主要分布					
		\begin{itemize}
			\item<1-> 东北平原
			\item<2-> 内蒙古高原
			\item<3-> 黄土高原
			\item<4-> 青藏高原
			\item<5-> 新疆山地
		\end{itemize}
	\end{columns}
\end{frame}

\subsection{研究现状}
\begin{frame}{\insertsection}{\insertsubsection}
	\begin{center}
		\includegraphics[width = 0.9\textwidth]{./pic/微生物养分和植物互相影响.jpg}
	\end{center}
\end{frame}
\begin{frame}{\insertsection}{\insertsubsection}
	\begin{columns}
		\column{0.5\textwidth}
		植物物种多样性与微生物多样性的关系	
		\begin{center}
			\includegraphics[width = 0.9\textwidth]{./pic/微生物养分和植物互相影响.jpg}
		\end{center}
		
		\column{0.5\textwidth}
		植物多样性通过几个方面对微生物产生直接或间接影响:
		\begin{itemize}
			\item<1-> 提高净初级生产力(NPP)
			\item<2-> 提高凋落物的多样性
			\item<3-> 提高食物资源和生境的多样性
			\item<4-> 改变凋落物的质量(C/N)
			\item<5-> 根分泌物的多样性
			\item<6-> 土壤温度、湿度、pH
		\end{itemize}
	\end{columns}
\end{frame}
\begin{frame}{\insertsection}{\insertsubsection}
	\begin{center}
		\includegraphics[width = 0.9\textwidth]{./pic/地下微生物群落.jpg}
	\end{center}
\end{frame}
\begin{frame}{\insertsection}{\insertsubsection}
	\begin{columns}
		\column{0.5\textwidth}
		地下微生物群落	
		\begin{center}
			\includegraphics[width = 0.9\textwidth]{./pic/地下微生物群落.jpg}
		\end{center}
		
		\column{0.5\textwidth}
		地下微生物群落在决定植物群落生产力、多样性和组成中发挥着重要作用(Kulmatiski et al., 2011)。前人的研究结果表明,地下微生物多样性会影响植物生长和生产力、有效养分、生态系统功能。还有一些研究显示微生物群落组成的变化会反馈植物的协同生存(Reynolds et al., 2003)和群落组成(Bever, 2003)。
	\end{columns}
\end{frame}




\subsection{研究现状}
\begin{frame}{\insertsection}{\insertsubsection}
\begin{center}
	\usetikzlibrary{shapes.geometric} % required in the preamble
	\smartdiagramset{
		uniform color list=gray!60!black for 6 items,
		font=\large,
		module minimum width=2.5cm,
		module minimum height=1.5cm,
		text width=1cm,
		circular distance=3cm,
		circular final arrow disabled=true,
	}
	\smartdiagram[circular diagram:clockwise]{植物多样性,冠层密度,土表蒸发,土壤水分,微生物多样性}
\end{center}
\end{frame}
\begin{frame}{\insertsection}{\insertsubsection}
	
\end{frame}
\begin{frame}{\insertsection}{\insertsubsection}
	
\end{frame}
\begin{frame}{\insertsection}{\insertsubsection}
	
\end{frame}


\subsection{研究目的}
\begin{frame}{\insertsection}{\insertsubsection}
	\begin{itemize}
		\item 沿水分梯度条件下,高寒草地和温性草地植物物种多样性和微生物多样性的关系
		\item 沿水分梯度条件下,高寒草地和温性草地植物功能群多样性与微生物多样性的关系
		\item 在不同生态系统类型条件下,植物物种多样性和功能群多样性对微生物多样性的影响
	\end{itemize}
\end{frame}
\section{研究方法、技术路线、实验方案}
\subsection{问题定义}
\begin{frame}{\insertsection}{\insertsubsection}
\end{frame}
\subsection{解决思路}
\begin{frame}{\insertsection}{\insertsubsection}
\end{frame}

\begin{frame}{\insertsection}{\insertsubsection}
	\begin{center}
		\usetikzlibrary{shapes.geometric} % required in the preamble
		\smartdiagramset{
			uniform color list=gray!60!black for 6 items,
			font=\large,
			module minimum width=4.5cm,
			module minimum height=1.5cm,
			text width=1cm,
			circular distance=5cm,
			circular final arrow disabled=true,
		}
		\smartdiagram[flow diagram]{冠层密度,土表蒸发,土壤水分,微生物多样性}
	\end{center}
\end{frame}

\section{研究方法、技术路线、实验方案}
\begin{frame}{\insertsection}
	\begin{center}
		\includegraphics[width = 0.9\textwidth]{./研究方法.jpg}
	\end{center}
\end{frame}

\subsection{研究方法}
\begin{frame}{\insertsection}{\insertsubsection}
	研究地点:
	于生长季顶期分别从西藏、青海、内蒙三个省份,沿从东到西的水分梯度分成南、北两条平行样带进行采样。记录每个样点的植被类型和温度,并用GPS记录每个样点的地理坐标和海拔。
\end{frame}
\subsection{技术路线}
\begin{frame}{\insertsection}{\insertsubsection}
	\begin{center}
		\includegraphics[width = 0.65\textwidth]{./技术路线.jpg}
	\end{center}
\end{frame}

\begin{frame}{\insertsection}{\insertsubsection}
\end{frame}
\subsection{实验方案}
\begin{frame}{\insertsection}{\insertsubsection}
\end{frame}

\begin{frame}{\insertsection}{\insertsubsection}
\end{frame}
\section{研究基础}
\
\subsection{问题定义}
\begin{frame}{\insertsection}{\insertsubsection}
\end{frame}
\subsection{解决思路}
\begin{frame}{\insertsection}{\insertsubsection}
\end{frame}

\begin{frame}{\insertsection}{\insertsubsection}
\end{frame}

\section{研究计划与工作安排}
\begin{frame}{\insertsection}{\insertsubsection}
\end{frame}
\begin{frame}{\insertsection}{\insertsubsection}
\end{frame}


%\ganttset{group/.append style={orange},
	milestone/.append style={black},
	progress label node anchor/.append style={text=red}}

\begin{ganttchart}[%Specs
	y unit title=0.5cm,
	y unit chart=0.5cm,
	vgrid,hgrid,
	title height=1,
	%     title/.style={fill=none},
	title label font=\bfseries\footnotesize,
	bar/.style={fill=gray},
	%  bar incomplete/.style={fill=white},
	bar height=0.68,
	progress label text={},
	group right shift=0,
	group top shift=0.7,
	group height=.5,
	group peaks width={0.5},
	inline]{1}{14}
	%labels
	\gantttitle{时间计划表}{14}\\  % title 1
	%%
	\gantttitle[]{2015--2016}{4}
	\gantttitle[]{2016--2017}{4}                 % title 2
	\gantttitle[]{2017--2018}{4} 
	\gantttitle[]{2018}{2}
	\\ 
	%%
	\gantttitle{Q1}{1}                      % title 3
	\gantttitle{Q2}{1}
	\gantttitle{Q3}{1}
	\gantttitle{Q4}{1}
	%%  
	\gantttitle{Q1}{1}                      % title 3
	\gantttitle{Q2}{1}
	\gantttitle{Q3}{1}
	\gantttitle{Q4}{1}
	%%   
	\gantttitle{Q1}{1}                      % title 3
	\gantttitle{Q2}{1}
	\gantttitle{Q3}{1}
	\gantttitle{Q4}{1}
	%%    
	\gantttitle{Q1}{1}                      % title 3
	\gantttitle{Q2}{1}
	% \gantttitle{Q3}{1}
	% \gantttitle{Q4}{1}
	\\
	%% 
	\ganttbar[progress=67,inline=false]{阅读文献}{1}{6}\\ 
	\ganttmilestone[inline=false]{\underline{\color{blue}开题报告}}{4}\\
	\ganttbar[progress=0,inline=false]{历史数据Meta分析}{5}{6}\\
	
	\ganttbar[progress=0,inline=false]{野外补样}{7}{7} \\
	%% 
	\ganttbar[progress=0,inline=false]{撰写植多样性与微生物多样性文章}{8}{10} \\
	\ganttmilestone[inline=false]{\underline{\color{blue}中期答辩}}{10}\\
	\ganttbar[progress=0,inline=false]{博士学位论文撰写}{11}{12} \\
	\ganttmilestone[inline=false]{\underline{\color{blue}毕业答辩}}{13}	
\end{ganttchart}
{\xdbg%末页致谢
\begin{frame}[plain,noframenumbering]
 \finalpage{{\huge 请批评指正!}}
\end{frame}}

%\finalpage{{\huge 感谢观看!}}

\end{document} 
